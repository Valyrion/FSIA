\documentclass[a4paper,12pt]{report}
\usepackage[utf8]{inputenc}
\usepackage[T1]{fontenc}
\usepackage{lmodern}
\usepackage[catalan]{babel}
\usepackage{mathtools}
\usepackage[ampersand]{easylist}
\begin{document}

\title{
	{\bf Filosofia del Software i Intel·ligència Artificial} \\ \vspace{2 mm}
	{\large Un anàlisi del passat, present i futur en el context tecnològic}
}
\author{
	Oriol Ventosa \and
	Marc Ferré \and
	Gonzalo Palacios \and
	Pol Gómez
}
\date{\today}
\maketitle

\tableofcontents

\chapter{Introducció}
\section{Ambicions}
Què esperem d'aquest projecte? Quin és el nostre objectiu, i potser més
important i tot; què esperem de tot això?


\begin{easylist}[itemize]
& Exposar les diferents filosofies que existeixen a l'hora de
	crear i distribuir \emph{software}.
& Aprofundir en els conceptes de \emph{llibertat} per part
	del consumidor de software.
& Proposar alternatives \emph{lliures} a software privatiu.
& Introduir conceptes comuns en l'àmbit matemàtic i tècnic
	de la intel·ligència artificial.
& Desenvolupar, amb un conjunt total d'eines lliures, una
	aplicació d'aprenentatge de màquina funcional.
\end{easylist}

\section{Eines}
A través del projecte de recerca, hem utilitzat diferents eines.
Ha prioritzat l'elecció de programari \emph{lliures} i \emph{open-source}.

\end{document}
