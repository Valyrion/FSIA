\section Introducció
La funció de l'intel·ligència artificial és fer accions que es consideren intel·ligents. Aquestes accions poden requerir robots, com ara construir la peça de un cotxe,
però d'altres només necessiten un programa informàtic, com per exemple un dels programes més utilitzats alhora de traduir textos a altres llengües: el google traductor.
Per tant la intel·ligència artificial intenta fer accions humanes de una manera intel·ligent i ,fins i tot, més eficient amb l'objectiu de millorar 
facilitar la vida a la raça humana.
Això desperta varies preguntes: Què es pot fer actualment amb l'intel·ligència artificial? Té futur o és simplement un projecte que porta a un carreró sense sortida? Pot
ser perillosa i , com a moltes pel·licules de ciència ficció passa, que ens intentin eliminar a la humanitat? Per què no es creen màquines intel·ligents de veritat?
Aquestes preguntes seràn contestades en aquest apartat.

\section Present de la IA
Actualment s'està treballant per crear intel·ligència artificial capaç de fer accions humanes com ara la visió o la manipulació d'objectes. Tot i que semblen accions facils
són molt dificils de implementar en programació perquè actuïn de manera intel·ligent. En el nostre dia a dia i han molts programes que utilitzen intel·ligència artificial:
google traductor, traductor instantani que haviem esmentat en la introducció, té una complexitat que no s'observa a simple vista ja que sembla molt fàcil buscar al diccionari
i buscar les paraules que et demanen, però no es té en compte que aquest programa té que dividir el text en parts i buscar les paraules adïents per a cada un dels casos seguint
les normes ortogràfiques i mantenint el significat del text original. En l'àmbit dels programes de l'intel·ligància artificial sense robòtica també podem incluir les recomenacions
de canals que t'ofereix youtube segons els teus gustos o les recomenacions de productes que moltes pàgines web, com ara ebay o amazon, t'ofereixen. També s'esta treballant creant
programes que et responen de manera llògica i raonada però moltes vegades acaba en fracàs i les seves respostes no tenen cap sentit, com pot ser el cas de cleverbot, o les seves
respostes són simples i pervisibles.
Pel que fà l'àmbit de la robòtica s'han creat robots capaços de caminar de manera òptima i evitar caigudes causats per els diferents tipus de terreny intentant simular animals com
ara gossos alhora de correr. També s'han creat humanoides, robots amb forma de persones, que són capaços de fer moviments humans, per exemple abraçar expressar emocions(no sentir-les),
manipular certs objectes... però fent accions de manera molt específica.
Un últim cas d'intel·ligència artificial és la que utilitzats maquinaria externa per fer funcionar un programa infomàtic. Aquest cas és els de reconeixement de veu natural ja que no
fà cap acció que es pugui veure a l'exterior però gràcies al micròfon que tenen fà que siguin capaces de analitzar la teva veu. Apple està experimentant amb aquest últim cas en els
seus nous productes amb un programa anomenat SIRI, un reconeixedor de veu que també és capaç de interpretar-la i respondret a la teva pregunta.

\section 
