\section{Definicions}
Abans de parlar sobre els tipus de llicencies software s'haurien de coneixer alguns conceptes bàsics.
\begin {itemize}
	\item \emph{Llicencies software:} són un contracte entre desarrollador del software, sotmès a propietat 	intelectual y els drets d'autor, i l'usuari. En aquest contracte es defineixen els drets i deures de 		ambdues parts. El desarrollador, o qui hagi cedit els drets d'explotació del producte, és la persona 		qui decideix quina llicencia software usar per la distribució del programa.
	\item \emph Patent: és el conjunt de drets exclusius concedits per un estat al creador o als creadors 		de un producte susceptible a ser explotat industrialitzalment, per un període limitat de temps a canvi 		de la divulgació de la incenció. Vol dir basicament que tercers no fagin ús de la tecnologia patentada.
	\item \emph{Drets d'autor o \texttit {copyright}:} és un conjunt de drets i normes que tenen els autors 	de creacions de obres de qualsevol tipus, tant cientifica, tecnològica, didàctica...
\end {itemize}
\section{Tipus de llicencies software}
Hi han molts tipus de llicencies software, com ara Academic Free Licence, Artistic Licence, Python Licence etc., però les més utilitzades són aquestes quatre:
\begin{itemize}
	\item \emph{}
