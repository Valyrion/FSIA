\section{Definicions}
Abans de parlar sobre els tipus de llicencies software s'haurien de coneixer alguns conceptes bàsics.
\begin {itemize}
	\item \emph{Llicencies software:} són un contracte entre desarrollador del software, sotmès a propietat 	intelectual y els drets d'autor, i l'usuari. En aquest contracte es defineixen els drets i deures de 		ambdues parts. El desarrollador, o qui hagi cedit els drets d'explotació del producte, és la persona 		qui decideix quina llicencia software usar per la distribució del programa.
	\item \emph Patent: és el conjunt de drets exclusius concedits per un estat al creador o als creadors 		de un producte susceptible a ser explotat industrialitzalment, per un període limitat de temps a canvi 		de la divulgació de la incenció. Vol dir basicament que tercers no fagin ús de la tecnologia patentada.
	\item \emph{Drets d'autor o \texttit {copyright}:} és un conjunt de drets i normes que tenen els autors 	de creacions de obres de qualsevol tipus, tant cientifica, tecnològica, didàctica...
\end {itemize}
\section{Tipus de llicencies software}
Hi han molts tipus de llicencies software, com ara Academic Free Licence, Artistic Licence, Python Licence etc., però les més utilitzades són aquestes quatre:
\begin{itemize}
	\item \emph{GPL:} prové de Llicencia Pública General GNU(GNU GPL). Aquesta llicència permet la copia, la distribució, tant amb fins comercial com no, i permet la modificació de aquest codi només si es segueix utilitzant el mateix tipus de llicencia GPL. Aquesta llicencia no permet executables sense mostrar el codi font d'aquesta. Aquest tipus de llicencia, la més usada en el món del software, garantitza a l'usuari final la llibertat de usar, estudiar, compartir i modificar el software amb el propòsit és evitar que el software tingui una llicencia de software lliure i protegir-lo dels intents d'apropiació que restringeixin les llibertats de l'usuari. Aquesta llicencia va ser creada per Richard Stallman, fundador de la Free Software Foundation per el projecte del grup GNU.
Segons aquest grup quan es parla de que és \textit free es refereixen a que és lliure no gratuït. Això vol dir que tu tens la llibertat de compartir i de modificar les versions del programa perquè estiguin segurs de que és lliure per tots el usuaris. Si fos gratis en comptes de lliure voldria dir que tu pots fer us del programa però no tindries el codi font per modificar-lo ni la llibertat per compartir-ho.

