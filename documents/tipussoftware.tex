\section{Definicions}
Abans de parlar sobre els tipus de llicencies software s'haurien de coneixer alguns conceptes bàsics.
\begin {itemize}
	\item \emph{Llicencies software:} són un contracte entre desarrollador del software, sotmès a 		propietat intelectual y els drets d'autor, i l'usuari. En aquest contracte es defineixen els drets i 		deures de ambdues parts. El desarrollador, o qui hagi cedit els drets d'explotació del producte, és la 		persona qui decideix quina llicencia software usar per la distribució del programa.
	\item \emph Patent: és el conjunt de drets exclusius concedits per un estat al creador o als creadors 		de un producte susceptible a ser explotat industrialitzalment, per un període limitat de temps a canvi 		de la divulgació de la incenció. Vol dir basicament que tercers no fagin ús de la tecnologia patentada.
	\item \emph{Drets d'autor o \texttit {copyright}:} és un conjunt de drets i normes que tenen els 		autors de creacions de obres de qualsevol tipus, tant cientifica, tecnològica, didàctica...
\end {itemize}
\section{Tipus de llicencies software}
Hi han molts tipus de llicencies software, com ara Academic Free Licence, Artistic Licence, Python Licence etc., però les més utilitzades són aquestes quatre:
\begin{itemize}
	\item \emph{GPL:} prové de Llicència Pública General GNU(GNU GPL). Aquesta llicència permet la copia, 		la distribució, tant amb fins comercial com no, i permet la modificació de aquest codi només si es 		segueix utilitzant el mateix tipus de llicencia GPL. Aquesta llicencia no permet executables sense 		mostrar el codi font d'aquesta. Aquest tipus de llicencia, la més usada en el món del software, 	garantitza a l'usuari final la llibertat de usar, estudiar, compartir i modificar el software amb el 		propòsit és evitar que el software tingui una llicencia de software lliure i protegir-lo dels intents 		d'apropiació que restringeixin les llibertats de l'usuari. Aquesta llicencia va ser creada per Richard 		Stallman, fundador de la Free Software Foundation per el projecte del grup GNU.
	Segons aquest grup quan es parla de que és \textit free es refereixen a que és lliure no gratuït. Això 		vol dir que tu tens la llibertat de compartir i de modificar les versions del programa perquè estiguin 		segurs	de que és lliure per tots el usuaris. Si fos gratis en comptes de lliure voldria dir que tu 	pots fer us del programa però no tindries el codi font per modificar-lo ni la llibertat per compartir-ho
	\item \emph{BSD:} o Berkeley Software Distribution és una llicencia software més permisiva que el GPL 		ja que aquesta té menys restriccions en comàració a la anterior que esta molt aprop del domini públic. 		La llicència BSD al contrari que la GPL pernet un ús del codi font en software no lliure. Aquesta 		llicència es podria definir com a molt liberal ja que no es fà resposable del que fas amb el teu 		software, o sigui que si per culpa teva es perden dades, es danyen ordinadors o obtens benefici per el 		teu producte, no et poden acusar. L'únic que has de tenir en compte per aquesta llicencia és mantenir 		el document de llicencia BSD. Molts sitemes operatius són descendents de BSD són SunOS, FreeBSD y MacOS X entre altres. També a aportat moltes millores en els sistemes operatius com ara el control de treballs o el Fast FileSystem, que permet una obtenció de dades més ràpida, entre d'altres
	\item \emph{MIT:} La llicència MIT (Massachusetts Institute of Technology) és molt similar a la llicència BSD ja que té unes carecterístiques molt similars: Tu pots fer el que vulguis amb el teu software mentre tu inclueixis el copyright inicial.

