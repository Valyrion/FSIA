\section{Organitzacions Defensores del Programari Lliure}

Per parlar d'empreses dedicades a el programari lliure, primer s'hauria de saber que és: Programari lliure.

El Programari Lliure és un afer de la llibertat dels usuaris per a executar, copiar, 		distribuir, canviar i millorar el programa en qüestió. Més precisament, es refereix a 4 "graus" de 		llibertat per als usuaris:

	\begin{itemize}
		\item La llibertat per a \textit{executar el programa}, per a qualsevol propòsit (grau  0).
		\item La llibertat \textit{estudiar} el funcionament del programa, \textit{adaptar-lo} a les 			necessitats pròpies del usuari i \textit{tenir accés al codi font} (grau 1).
		\item La llibertat poder \textit{distribuir} copies amb una comunitat (grau 2).
		\item La llibertat \textit{configurar} d'una manera beneficiosa el programa, i tenir la 		possibilitat de	distribuir copies d'aquesta millora a una communitat per a que es puguin 			beneficiar. (grau 3)
	\end{itemize}
Les organitzacions més influents que s'encarreguen de defensar el software lliure són:
 
	\emph{Electronic Frontier Foundation}: És una organitzacio sense ànim de lucre basada en part 		en la primera enmenda de la Constitució d'Estats Units, que, defensa la llibertat d'expressió, l'únic 		que aquesta defensa els ciberdrets. Formada en 1990 per \textit{Mitch Kapor, John Gilmore i John 		Perry}. Com a organització lliure volen garantir "d'una manera diferent" els graus de llibertat dels 		\emph{bloggers} intentant garantir la màxima llibertat per \textit{expressar idees de manera anònima i 		amb uns certs drets.}
	\emph{Free Software Foundation}: Al igual que \emph{Electronic Frontier Foundation}, és una 		organització sense ànim de lucre fundada per R. Stallman. És, possiblement la organització més 		influent del programari lliure, format alhora per una comunitat ètica en tot el món dedicada 		exclusivament a el software lliure i la distribució d'aquest que produeix multiples tasques, tals com:
		
	\begin{itemize}
	\item Mantenir la definició de programari lliure
	\item Mantenir una educació legal, celebrant sovint seminaris sobre aspectes legals de fer servir la 		llicència \textbf{GPL} (resumidament, és una llicència que garantitza als usuaris els 4 graus de 		llibertat) i ofereix un servei de consulta per a advocats.
	\item Conseguir que tothom tingui la possibilitat de \textbf{tindre control sobre la tecnologia} 		quotidiana, sense restriccions governamentals i només per aconseguir un benefici individual o 		communitari. El projecte que vol obtenir aquest objectiu està en desenvolupament i s'anomena 		\textbf{GNU}.
	\end{itemize}

\section{Casos d'èxit de Software Lliure}
	\emph{Apple Inc.}: Apple és una gran empresa, que va començar amb unes expectatives baixes però 		actualment és una de les empreses més grans del món, actualment no és tot lliure però comença a 	decantar tota la balança cap al \emph{software lliure}, i és que proximament Apple ha decidit 		donar el seu codi font de manera gratuïta i seva molt esperada actualització a Mac OS X (Mavericks), 		juntament amb altres programes com \emph{iLife} i \emph{iWork}. La estrategia en qüestió és, 		proporcionarte de la manera més rapida i fàcil possible la última actualització.

	\emph{Apache OpenOffice}: És un \emph{"paquet ofimatic"}, és a dir, un conjunt de programes per 		a oficina similars als de Microsoft Office però de manera gratuïta i basat en Java. Traduït a més de 		110 idiomes i amb un assistent per instal·lar extensions amb la qual és possible descarregar 		diccionaris addicionals per a molts idiomes.

	\emph{Adobe Acrobat}: \emph{Adobe Acrobat} és una família de programari d'aplicacions i serveis 		web desenvolupats per \emph{Adobe Systems} per veure, crear, manipular, imprimir i gestionar arxius en 		format PDF (\emph{Portable Document Format}). El que permet al usuari manipular el codi font és 	JavaScript.
	

\section{Bibliografia}

\begin{itemize}
\item{http://ca.wikipedia.org/wiki/Electronic\_Frontier\_Foundation}
\item{https://www.eff.org/about}
\item{https://www.eff.org/bloggers}
\item{http://www.gnu.org/philosophy/philosophy.html}
\item{http://john.do/blog-blogger-blogging/}
\item{http://www.fsf.org/appeal/2009/freedom-is-the-goal}
\item{http://www.businessweek.com/articles/2013-10-22/why-apple-wants-its-software-to-be-free}
\item{http://www.adobe.com/accessibility/products/reader.html}
\item{http://ca.wikipedia.org/wiki/Adobe\_Acrobat}
\end{itemize}






