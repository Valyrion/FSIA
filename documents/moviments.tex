\section{Organitzacions Defensores del Programari Lliure}

Per parlar d'empreses dedicades a el programari lliure, primer s'hauria de saber que és: Programari lliure i de Propietat.

El Programari de Propietat o Privatiu és aquell que té restriccions en l'ús, publicació de versions modificades o no modificades. Usualment el codi font no és públic (no visible als usuaris). En el cas de que el codi font sigui públic, NO té per que ser lliure si mantenen les restriccions anteriors. \cite{ProgPro}

El Programari Lliure és un afer de la llibertat dels usuaris per a executar, copiar, distribuir, canviar i mllorar el programa en qüestió. Més precisament, es refereix a 4 "graus" de llibertat per als usuaris:

	\begin{itemize}
		\item La llibertat per a \textbf{executar el programa}, per a qualsevol propòsit (grau  0).
		\item La llibertat \textbf{estudiar} el funcionament del programa, \textbf{adaptar-lo} a les 			necessitats pròpies del usuari i \textbf{tenir accés al codi font} (grau 1).
		\item La llibertat poder \textit{distribuir} copies amb una comunitat (grau 2).
		\item La llibertat \textbf{configurar} d'una manera beneficiosa el programa, i tenir la 		possibilitat de	distribuir copies d'aquesta millora a una communitat per a que es puguin 			beneficiar. (grau 3)
	\end{itemize}

Les organitzacions més influents que s'encarreguen de defensar el software lliure són: 
 
\textbf{\emph{Electronic Frontier Foundation}}: És una organitzacio sense ànim de lucre basada en part en la primera enmenda de la Constitució d'Estats Units, que, defensa la llibertat d'expressió, l'únic que aquesta defensa els ciberdrets. Formada en 1990 per \textit{Mitch Kapor, John Gilmore i John Perry}. Com a organització lliure volen garantir "d'una manera diferent" els graus de llibertat dels \emph{bloggers} intentant garantir la màxima llibertat per \textit{expressar idees de manera anònima i amb uns certs drets.} \cite{OrgDefEFF}
 \cite{OrgDefEFFII}
\textbf{\emph{Free Software Foundation}}: Al igual que \emph{Electronic Frontier Foundation}, és una 		organització sense ànim de lucre fundada per R. Stallman. És, possiblement la organització més 		influent del programari lliure, format alhora per una comunitat ètica en tot el món dedicada 		exclusivament a el software lliure i la distribució d'aquest que produeix multiples tasques, tals com:

	\begin{itemize}
	\item Mantenir la definició de programari lliure
	\item Mantenir una educació legal, celebrant sovint seminaris sobre aspectes legals de fer servir la 		llicència \textbf{GPL} (resumidament, és una llicència que garantitza als usuaris els 4 graus de 		llibertat) i ofereix un servei de consulta per a advocats.
	\item Conseguir que tothom tingui la possibilitat de \textbf{tindre control sobre la tecnologia} 		quotidiana, sense restriccions governamentals i només per aconseguir un benefici individual o 		communitari. El projecte que vol obtenir aquest objectiu està en desenvolupament i s'anomena 		\textbf{GNU}. \cite{ObjGNU} \cite{OrgDefFSF}
	\end{itemize}

\section{Casos d'èxit de Software Lliure}

\begin{itemize}

\item \textbf{\emph{GNU}}: GNU va ser creat per \textit{Richard Stallman} el 1983, com a un sistema operatiu que es va posar en marcha per les persones que treballaven i segueixen treballant juntes per la llibertat de tots els usuaris del programari per poder gaudir de tots els graus de llibertat d'una manera \textit{total}. La intenció di filosofia de GNU es basa en, NO \textit{boicotejar} els programes amb software privatiu i, expliquen que rebutjar un prgrama que ens perjudica, no es boicotejar, és racionalitat comúna. Les motivacions principals que van portar a Richard a duur a terme GNU estàn recollides en un document escrit per ell anomenat:\texit{Manifest GNU} \cite{GNUExit} \cite{GNUExitII} \cite{GNUMan} \cite{GvsM}

\item \textbf{\emph{Mozilla Corporation}}: Empresa filial propietaria total de la \emph{Fundació Mozilla}, sense ànim de lucre, coordinadora i responsable de l'integrament de aplicacions informatiques tals com el conegut navegador web \textit{Mozilla Firefox} o el client de correu electònic \textit{Mozilla ThunderBird}. Aquests programes es poleixen diariament per mitja de millons de programadors voluntaris que treballen juntament amb els de la corporació que es regeixen per uns principis que l'empresa té establits; el document que els conté s'anomena \emph{MANIFESTO} per un bé comú: que tothom gaudeixi de la llibertat. \cite{MozExit} \cite{MozExitII} \cite{MozFesto}

\item \textbf{\emph{Linux}}: \emph{Linux}: Linux és un sistema operatiu lliure, per tant al obtenir aquest obtens el codi font. Dissenyat per millons de programadors de tot el món, i actualment segueix en desenvolupament sota la coordinació de \emph{Linus Torvalds}.Cada dia s'afeieixen nous continguts, com programes que venen distribuits sota la llicencia de \emph{GNU}.Linux ofereix características estándar de Unix, com el support multi-usuari, multitasca, creació de reds i el cumpliment de \emph{POSIX}(és l'acrònim de \emph{Portable Operating System Interface} l'última sigla fa referencia a UNIX). \cite{LinExit} \cite{POSIX}

\item \textbf{\emph{Chromium}}: \emph{Chromium} és un projecte de navegador de codi obert que té com a objectiu construir una manera més segura, més ràpida i estable per a els usuaris d'experimentar amb internet. El navegador conté documents de disseny, informació de proves i altres continguts per ajudar a aprendre, construir i treballar amb el seu codi font. També es la base de \textit{Google Chrome}. \cite{Chrom} 

\end{itemize}





\section{Bibliografia}

\begin{itemize}
\item{http://ca.wikipedia.org/wiki/Electronic\_Frontier\_Foundation}
\item{https://www.eff.org/about}
\item{https://www.eff.org/bloggers}
\item{http://www.gnu.org/philosophy/philosophy.html}
\item{http://john.do/blog-blogger-blogging/}
\item{http://www.fsf.org/appeal/2009/freedom-is-the-goal}


\end{itemize}






