\section{Introducció}

El concepte d'una "màquina pensant" va començar el 2500 a.C, quan els egipcis miraven a estàtues "parlants" en busca de consells místics. Époques després, durant el segle XV, els automats preferits de al societat eren els óssos que tocaven tambors, figuretes ballarines que apareixien cada vegada que un rellotge marcava l'hora i l'\textbf{"autòmat"} dissenyat per en Wolfgang von Kempelen, una maquina invencible en els escacs que va regnar durant el segle XVIII. Isaac Asimov, un simbol en el camp de la robòtica, va ser escriptor, erudit i autor de les lleis de la robòtica. Asimov estaba anys llum dels pensadors de l'època i va fer prediccions en les quals la "cibernètica" (a Asimov li agradava refereir-se a la robòtica amb el nom de cibernètica), provocaria una revolució intel·lectual. 
Issac Asimov va escriure en el pròleg de "Thinking by Machine", de Pierre de Latil:

-La cibernètica no és merament una branca de la ciència, és una revolució intel·lectual que rivalitza en importància la Revolució Industrial. És possible que al igual que una màquina pot fer-se càrrec de les funcions rutinàries dels músculs humans, un altre pugui fer-se càrrec dels usos de rutinaris de la ment humana?
La c

Finalment el terme va ser inventat el 1956, a \textbf{la Conferència de Darmouth}, un congrés en el qual es van fer previsions triomfalistes a deu anys que mai es van complir, el que va provocar l'abandó gairebé total de les investigacions durant quinze anys. La Conferencia de Darmouth va intentar esbrinar com fabricar màquines que utilitzin el llenguatge, formin abstraccions i conceptes i siguin capaçes de auto-millorar-se; l'estudi va durar 2 mesos i estava format per 10 persones. Això sense contar que el matemàtic, lògic, científic de la computació, criptògraf  i filòsof britànic \textbf{Alan Turing} ja havia dissenyat en el 1936-1937 la \textbf{\emph{Maquina de Turing}}. La pregunta bàsica que Turing va tractar de respondre era:  \cite{Matur}

\textbf{-Poden les maquines pensar?}

Els arguments a favor de Turing sobre la intel·ligencia artificial, van iniciar un debat intens que va marcar clarament la primera etapa de interacció entre la intel·ligencia artificial i psicologia. De fet se sap que anys enrere diferents filosofs i matematics ja hi pensaben: tant en la intel·ligencia artificial com en el seu funcionament; per exemple, Aristoteles, va ser el primer en descriure de manera estructurada un conjunt de regles que describien el funcionament de la ment humana. \cite{IAgen} \cite{IAgenII}

\textbf{Peró, que és la maquina de Turing?}

Doncs bé, la maquina de Turing va ser dissenyada principalment per a implementar qualsevol problema per mitja de algoritmes.\cite{Algor}

\section{Evolució i poliment de la IA}
\begin{itemize}
\item  \textbf{Aristóteles} (300 a.C): Va ser el primer en descriure de manera estructurada un conjunt de regles que describien el funcionament de la ment humana
\item  \textbf{Ctesibio de Alejandría} (250 a.C): Dessarrola una maquina capaç de regular el fluxe d'aigua que actua modificant el comportament de la màquina.
\item \textbf{Gottlob Frege} (1879): Amplia la lógica booleana i obté la Logica del Primer Ordre (sistema lògic-deductiu i que restringeix quines són les expressions correctament formades). Aquesta ordre és tant sumament important que té el poder expressiu suficient per definir pràcticament totes les matemàtiques.
\item \textbf{Lee De Forest} (1903): Inventa el tríode, un component electrònic usat per amplificar, commutar, o modificar una senyal elèctrica. \cite{Tri}
\item \textbf{Alan Turing} (1936-1937): Considerat com el pare de la ciència informàtica, va formular el concepte de \emph{algoritme}, inventor de la \emph{Maquina de Turing}, va ajudar a Inglaterra contra els Alemans en la primera guerra mundial i va publicar un article on és va demostrar que existeixen problemes dels quals no es pot obtenir una solució; ni per mitja humà ni per l'ús de computadores.
\item \textbf{Warren McCulloch i Walter Pitts} (1943): Van formular un model de neurones artificials sense considerar-se un treball del camp de la intel·ligencia artificial degut a la inexistència d'aquesta en la època.\cite{EvoIA}
\end{itemize} 

\section{El test de Turing}

\emph{Alan Turing}, apart de ser un inventor i persona d'èxit, va dissenyar un simple i lògic sistema durant el1950 capaç de \textbf{verificar si una maquina és o no intel·ligent.}

\textbf{En que és basa?}

El test és duu a terme simplement situant un humà i una maquina separats per una paret. EL test es basara en l'humà, que haurà de anar formulant preguntes i la maquina, que les haurà de respondre. L'humà, al ser inconsient de que està parlant amb una maquina, haurà de jutjar si les respostes que rep són lògiques, o no, si l'humà creu que si ho són, es considerarà que la maquina en qüestió és intel·ligent. A internet hi han \textbf{moltes variacions de aquest test}, però de fet, aquest \textbf{és el original.} \cite{TurTest} 

\section {Fets Curiosos al llarg de la historia de la IA}

\begin{itemize}
\item L'autòmat inventat per en Wolfgang von Kempelen va resultar ser una estafa ja que en l'interior d'aquesta hi actuava un jugador d'escacs professional.
\item La gent acostumaba a imaginar que per allà l'any 1984 la nostra vida es veuria dominada per la tecnologia, servits per robots, cases
\item "Què és l'intel·ligència artifacial?", li preguntes a Google. Al que ell respon: "Et refereixes a la intel·ligència artificial?" Per descomptat que si. Mentrestant, en els teus 0,15 segons que t'han portat a reconeixer la teva estupidesa, una màquina intel·ligent ha reunit 17.900.000 resultats per a la teva consideració. \cite{InterFacts}
\end{itemize}



