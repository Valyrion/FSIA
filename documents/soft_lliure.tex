\section{Què és?}

El software lliure (de l'anglès \emph{"free software"}) és tot aquell software publicat
sota llicències que respecten el concepte de \emph{llibertat}. Degut a que la definició
(en relació al software) de llibertat és molt àmplia, i es podria dedicar inacabable espai
a definir-la, utilitzarem la definició més acceptada, la que ha popularitzat la \ac{fsf}; un programa és lliure si al adquirir-lo, l'usuari pot fer-lo servir,
copiar-lo, estudiar-lo, modificar-lo, i redistribuir-lo lliurement de diferents formes. \cite{wikifree}

Que el software sigui lliure, no implica que el seu cost sigui zero (encara que molt
software lliure, també sigui gratuït) \cite{sellingfree}. Per tant, jo puc crear software lliure i distribuir els arxius
binaris a \EUR{500,000}, mentre mantingui un lliure accés al codi font. El desenvolupament de software
lliure es sol finançar a través de donacions voluntàries, i la major part de projectes dediquen
aquests diners recaptats al manteniment dels serveis que el software que distribueixen necessita
(espai web, e.g).

\section{Qui el fa?}

La major part de software lliure no és realitzat per companyies multimilionàries; en canvi,
petits i mitjans grups de desenvolupadors són qui donen vida al software lliure.
De totes formes, la major part de projectes de software lliure relativament importants
són suportats i actualitzats per empreses que no es dediquen exclusivament a \emph{crear}
software lliure, però que n'utilitzen, i això les motiva a millorar-lo.

\section{Història}

Des dels anys 50 fins als 70, no hi havien grans corporacions que 
llicenciessin software, i es solia compartir de forma lliure entre
programadors, i distribuir de forma integrada en el \emph{hardware}
(es a dir, els ordinadors). Una vegada entrats els 70, la indústria del
software va començar a mostrar la seva capacitat econòmica, i es va
començar a vendre programari per separat. \cite{ibmusdata}

En \emph{Richard Matthew Stallman}, va anunciar el projecte \emph{GNU} (\emph{GNU no és Unix!}, sistema operatiu lliure), argumentant que s'havia cansat dels efectes del canvi en la cultura de la indústria informàtica i els seus usuaris. La \ac{fsf} va ser fundada l'Octubre de 1984. Va desenvolupar una de les definicions de \emph{software lliure} més acceptades, i el concepte de \emph{copyleft}, dissenyat per a assegurar la llibertat de software per a tothom. \cite{fossieee}.

A partir d'aquell moment, la comunitat de software lliure va començar a créixer de forma estable.

\section{Us actual}

En l'actualitat, hi ha molt software lliure en circulació activa, i una gran comunitat de desenvolupadors darrere d'ell, però no és utilitzat de forma tan estesa com el software propietari.

No només els usuaris particulars tenen a l'abast (i utilitzen) software lliure: molts governs han fet (o estan fent) el traspàs al software d'aquest tipus (Kerala, a la Índia, Munich, a Alemanya, Veneçuela, Malàisia, Perú, Equador, i altres). \cite{fossadopters}

\section{Avantatges i inconvenients}

El software lliure té una quantitat important d'avantatges  \cite{fossadvantages}:

\begin{enumerate}
\item \emph{Econòmic} - molta part del FOSS (software de codi obert i lliure) és gratuït, o té un preu molt baix. Les petites empreses es poden beneficiar d'això, i expandir la seva infraestructura informàtica sense gastar milers d'euros en software propietari.
\item \emph{Llibertat d'ús i distribució} - es pot instal·lar software lliure sense limitacions per culpa de llicències d'un sol ús, com passa amb els sistemes operatius i 'suites' ofimàtiques.
\item \emph{Independència tecnològica} - l'accés al codi font permet desenvolupar nous productes amb una base sòlida de software, o ajustar-lo a les nostres necessitats. Quan s'utilitza software lliure, no s'ha de patir per les decisions de l'entitat creadora, ja que sempre es tindrà accés a versions més antigues, o en casos més avançats, al codi font.
\item \emph{Sistemes sense 'backdoors'} - tenir accés al codi impossibilita la implementació de \emph{espies} en el codi font d'un programa.
\item \emph{Correcció més ràpida i eficient d'errors} - la comunitat activa de desenvolupadors de software lliure realitzen actualitzacions constants, i errors o 'bugs' són arreglats molt més ràpid que en el software propietari.
\end{enumerate}

De totes formes, també té desavantatges \cite{gentegeek}:

\begin{enumerate}
\item \emph{Falta de garantia} - el software lliure no ofereix garanties; si es trenca, no és culpa de ningú excepte teva, si es que has fet alguna cosa malament.
\item \emph{Difícil d'adquirir} - hi ha una certa quantitat de software lliure que és més complicat de descarregar i instal·lar en comparació amb el seu cosí propietari. Es deu, en molts casos, en que els autors es centren en el \emph{codi} del programa, i deixen a responsabilitat de l'usuari la tasca de compilació i instal·lació del programari.
\item \emph{Acabat final} - molt software lliure ofereix un aspecte gràfic o final poc atractiu; no hi han equips dedicats íntegrament al desenvolupament de l'interfície gràfica, i es fa el que es pot, amb els recursos que es tenen.
\item \emph{Entreteniment} - els títols \emph{AAA} (jocs amb pressupost molt elevat) no són lliures. Falta molt de temps per a que l'usuari comú pugui observar l'impressionant codi font de títols com \emph{Battlefield} o \emph{Far Cry}, ja que el mercat és més lucratiu que ètic, i s'hi troben molts inconvenients en alliberar el codi font d'un \emph{engine} (motor gràfic).
\end{enumerate}