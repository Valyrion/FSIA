Aquest projecte de recerca presenta dos temes principals, un apartat essent els
\emph{conceptes} i l'altre els \emph{procediments}.

El \emph{software}, i les seves filosofies i llicències formaran el primer i més
teòric apartat. S'explicarà què és el software, i es separarà en \emph{privatiu}
i \emph{lliure}, dues formes de veure la creació i distribució del matex. A més,
s'introduirà l'intel·ligència artificial, la seva història, el seu present i el
seu hipotètic futur, com a preludi pels procediments del projecte.

El segon apartat, sobre \emph{intel·ligència artificial}, ens guiará pel procés
que s'ha seguit per a crear software intel·ligent, capaç d'aprendre constantment
i de preveure el futur, fins a un producte finalitzat, i que combina amb eloqüència
ambdós apartats del projecte, per a finalitzar el treball.

Amb aquest projecte es pretén trobar un punt en comú entre les matemàtiques (que
formen la base teòrica de la IA), la intel·ligència artificial, i el software
(específicament, lliure).

S'ha treballat de forma altament col·laborativa, a través d'una plataforma
on-line anomenada \href{http://github.com}{GitHub}, que implementant un sistema
de control de versions anomenat \href{http://git-scm.com/}{Git} descentralitza
el projecte, i permetre el lliure accés a col·laboradors i, especialment, membres
del mateix grup.
