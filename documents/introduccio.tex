Aquest projecte de recerca es divideix en dos seccions; els \emph{conceptes}
i els \emph{procediments}.

El \emph{software}, i les seves filosofies i llicències formaran el primer i més
teòric apartat. S'explicarà què és el software, i es separarà en \emph{propietari}
i \emph{lliure}, dues formes de veure la creació i distribució del mateix. A més,
s'introduirà l'\emph{intel·ligència artificial} (IA), la seva història, el seu present i el
seu hipotètic futur, com a preludi pels procediments del projecte.

El segon apartat, sobre \emph{intel·ligència artificial}, s'explicarà el procés
que s'ha seguit per a crear un seguit de \emph{demostracions}, que apliquen un
dels mètodes més coneguts de la IA; les xarxes neuronals. A través de demostracions
gràfiques es podrà entendre millor en concepte de les ANN (\emph{Artificial Neural Networks}).

Amb aquest projecte es pretén trobar un punt en comú entre les matemàtiques,
la intel·ligència artificial, i el software (específicament, lliure).

S'ha treballat de forma altament col·laborativa, a través d'una plataforma
en línia anomenada \href{http://github.com}{GitHub}, que implementant un sistema
de control de versions anomenat \href{http://git-scm.com/}{Git} descentralitza
el projecte, i permet el lliure accés a col·laboradors externs i, especialment, a membres
del mateix grup.