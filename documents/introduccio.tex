\section{Objectius}

Aquest projecte combina dos móns que semblen molt separats per definició, el
\emph{software} i la \emph{intel·ligència artificial} (IA). Tot i que és cert
que no són àmbits dependents entre sí, es troben en un punt en comú;
el moment en que la IA es proposa saltar de l'abstracció
purament matemàtica a la funció real, en un món real.

Aquí és on col·lisionen els dos móns, i on entrem en joc nosaltres. El
nostre seguit d'objectius, en l'apartat de software és el següent:


\begin{itemize}
	\item Exposar les diferents filosofies que existeixen a l'hora de
	crear i distribuir software.
	\item Aprofundir en el concepte de llibertat per part
	del consumidor de software.
	\item Proposar alternatives lliures a software privatiu.
	\item Mostrar altres contextos en el que l'usuari no és lliure.
\end{itemize}

I en la segona part del projecte, en l'apartat d'intel·ligència artificial:

\begin{itemize}
	\item Introduir conceptes comuns en l'àmbit matemàtic i tècnic
	de la intel·ligència artificial.
	\item Desenvolupar, amb un conjunt total d'eines lliures, una
	aplicació d'aprenentatge de màquina funcional.
	\item Realitzar una mirada al present, veure què hi ha en
	desenvolupament.
	\item Realitzar una mirada al futur, veure què podem esperar
	en el nostre temps de vida.
\end{itemize}
