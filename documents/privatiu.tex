\section{Què és?}

Anomenem software privatiu a tot aquell programa amparat sota llicències
que reserven un o tots els drets d'ús, còpia, modificació i distribució
al fabricant qui, pagant, concedeix un ús del programa executable al titular
de la llicència, és a dir, és l'antònim del software lliure. Tal i com el 
seu nom indica, és software privat. Generalment, per a protegir les dades 
sobre el programa, el seu codi font, és a dir, el llenguatge de programació 
amb el que ha estat escrit originalment abans de ser compilat, no és visible, 
ja que la gent podria mirar-lo analitzar-lo i copiar-lo o bé modificar-lo 
i revendre'l o, directament, regalar-lo. A més són programes llargs i complexos
 que per a descobrir-los requereixen un temps d'estudi considerable i un procés d'enginyeria inversa llarg i costós.

\section{Qui el fa?}

El principal desenvolupador de software privatiu a nivell mundial és Microsoft, 
tot i així també trobem extensions javascript que entren als nostres navegadors 
i són privatives, però el percentatge de extensions privatives no és excessiu 
tampoc ens podem oblidar del buscador més utilitzat a nivell mundial, 
\emph{Google}.  Si el voleu evitar és recomenable buscar amb certs buscadors concrets 
com \emph{DuckDuckGo}. També fa ús d'aquest software privatiu i la famosa empresa
\emph{Apple}.

\section{Ús actual}

Avui en dia el software privatiu és utilitzat en gairebé el 99\% dels programes que
podem trobar a internet. Com ja hem anomenat abans \emph{Google, Apple, Microsoft}...
i la majoria de webs populars.

\section{Avantatges de la privacitat}

La privacitat, en certa manera és bona si el que vols és protegir els interessos de
la teva creació, si el que vols és que una web o un software et reporti interessos 
i el fas en codi lliure, obviament, els interessos seran més baixos ja que la gent 
el podrà copiar o descarregar-lo sense pagar, mentre que amb el privatiu ningú el
copiarà i protegirà així la propietat intel·letual del creador.

\section{Bibliografia}

http://www.gnu.org/philosophy/microsoft.ca.html

http://ca.wikipedia.org/wiki/Programari\_de\_propietat

http://www.gnu.org/philosophy/categories.ca.html

http://javinforever.wordpress.com/2012/02/21/software-lliure-i-software-privatiu/

