\section{Què és?}

Anomenem software privatiu a tot aquell programa emparat sota llicències
que reserven un o tots els drets d'ús, còpia, modificació i distribució
al fabricant qui, pagant, concedeix un ús del programa executable al titular
de la llicència. És a dir, és l'antònim del software lliure.

Generalment, per a protegir les dades sobre el programa, el seu codi font
(codi que estipula com funciona el programa), no és visible a tothom, ja 
que la gent podria mirar-lo, analitzar-lo i copiar-lo o bé modificar-lo i
revendre'l o, directament, regalar-lo. \cite{wikipediapropietari}\cite{gnucategories}

\section{Qui el fa?}

El principal desenvolupador de software privatiu a nivell mundial és Microsoft
\cite{gnumicrosoft}, 
També fa ús d'aquest software privatiu, la famosa empresa
\emph{Apple}.

\section{Ús actual}

Avui en dia el software privatiu és utilitzat en gairebé el 99\% dels programes que
podem trobar a internet. Com ja hem anomenat abans \emph{Google, Apple, Microsoft}...
i la majoria de webs populars. \cite{gnumicrosoft}

\section{Avantatges de la privacitat}

La privacitat, en certa manera és bona si el que vols és protegir els interessos de
la teva creació, si el que vols és que una web o un software et reporti interessos 
i el fas en codi lliure, obviament, els interessos seran més baixos ja que la gent 
el podrà copiar o descarregar-lo sense pagar, mentre que amb el privatiu ningú el
copiarà i protegirà així la propietat intel·letual del creador.


