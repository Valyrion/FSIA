\section{Què és?}

Anomenem software privatiu a tot aquell programa publicat sota llicències
que reserven un o tots els drets d'ús, còpia, modificació i distribució
al fabricant qui, pagant, concedeix un ús del programa executable al titular
de la llicència.

Generalment, per a protegir les dades sobre el programa, el seu codi font
(codi que estipula com funciona el programa), no és visible a tothom, ja 
que la gent podria mirar-lo, analitzar-lo i copiar-lo o bé modificar-lo i
revendre'l o, directament, regalar-lo. \cite{wikipediapropietari}\cite{gnucategories}

\section{Qui el fa?}

El principal desenvolupador de software privatiu a nivell mundial és Microsoft
\cite{gnumicrosoft}, 
També fa ús d'aquest software privatiu, la famosa empresa \emph{Apple, Oracle, Adobe,
VMware...}

\section{Ús actual}

Avui en dia molta part del software utilitzat per la majoría de població, és privatiu.

Aquesta gran extensió del seu ús és degut a l'inversió millonària al màrketing, i a
pactes amb productors de sistemes operatius i proveidors d'Internet, que acorden la
prèvia instal·lació de software privatiu als ordinadors.

ESTADÍSTIQUES D'ÚS

\section{Avantatges de la privacitat}

REFER


