\section {Historia de la IA}

El terme va ser inventat el 1956, a \textbf{la Conferència de Darmouth}, un congrés en el qual es van fer previsions triomfalistes a deu anys que mai es van complir, el que va provocar l'abandó gairebé total de les investigacions durant quinze anys. La Conferencia de Darmouth va intentar esbrinar com fabricar màquines que utilitzin el llenguatge, formin abstraccions i conceptes i siguin capaçes de auto-millorar-se; l'estudi va durar 2 mesos i estava format per 10 persones. Això sense contar que el matemàtic, lògic, científic de la computació, criptògraf  i filòsof britànic \textbf{Alan Turing} ja havia dissenyat en el 1936-1937 la \textbf{\emph{Maquina de Turing}}. La pregunta bàsica Turing va tractar de respondre era:
\begin{itemize} 
\item \textbf{Poden les maquines pensar?}
\end{itemize}
Els arguments a favor de Turing sobre la intel·ligencia artificial, van iniciar un debat intens que va marcar clarament la primera etapa de interacció entre la intel·ligencia artificial i psicologia. De fet se sap que anys enrere diferents filosofs i matematics ja hi pensaben: tant en la intel·ligencia artificial com en el seu funcionament; per exemple, Aristoteles, va ser el primer en descriure de manera estructurada un conjunt de regles que describien el funcionament de la ment humana.

Peró, que és la maquina de Turing?

Doncs bé, la maquina de Turing va ser dissenyada principalment per a implementar qualsevol problema per mitja de algoritmes.\cite {algoritme cite}

\section{Evolució i poliment de la IA}
\begin{itemize}
\item  \textbf{Aristóteles} (300 a.C): Va ser el primer en descriure de manera estructurada un conjunt de regles que describien el funcionament de la ment humana
\item  \textbf{Ctesibio de Alejandría} (250 a.C): Dessarrola una maquina capaç de regular el fluxe d'aigua que actua modificant el comportament de la màquina.
\tiem \textbf{Gottlob Frege} (1879): Amplia la lógica booleana i obté la Logica del Primer Ordre (sistema lògic-deductiu i que restringeix quines són les expressions correctament formades). Aquesta ordre és tant sumament important que té el poder expressiu suficient per definir pràcticament totes les matemàtiques.
\item \textbf{Lee De Forest} (1903): Inventa el tríode, un component electrònic usat per amplificar, commutar, o modificar una senyal elèctrica.
\item \textbf{Alan Turing} (1936-1937): Considerat com el pare de la ciència informàtica, va formular el concepte de \emph{algoritme}, inventor de la \emph{Maquina de Turing}, va ajudar a Englaterra contra els Alemans en la primera guerra mundial i va publicar un article on és va demostrar que existeixen problemes dels quals no es pot obtenir una solució; ni per mitja humà ni per l'ús de computadores.
\item \textbf{Warren McCulloch i Walter Pitts} (1943): Van formular un model de neurones artificials sense considerar-se un treball del camp de la intel·ligencia artificial degut a la inexistència d'aquesta en la època.
\end{itemize}

\section{El test de Turing}

\emph{Alan Turing}, apart de ser un inventor i persona d'èxit, va dissenyar un simple i lògic sistema durant el1950 capaç de \textbf{verificar si una maquina és o no intel·ligent.}

\section{En que és basa?}

Eltest és simplement situar un humà i una maquina separats per una paret. EL test es basara en l'humà, que haurà de anar formulant preguntes i la maquina, que les haurà de respondre. L'humà, al ser inconsient de que està parlant amb una maquina, haurà de jutjar si les respostes que rep són lògiques, o no. 




\section{Bibliografia}

http://www.slideshare.net/cchicaiza/mquina-de-turing
http://answers.yahoo.com/question/index?qid=20090522102609AAVjnYz {expansió i info sobre: Que es un algoritme?}
http://es.wikipedia.org/wiki/Historia_de_la_inteligencia_artificial#Antecedentes
http://ingeniatic.euitt.upm.es/index.php/tecnologias/item/637-triodo
http://www.dma.eui.upm.es/historia_informatica/Doc/Personajes/AlanTuring.htm


























